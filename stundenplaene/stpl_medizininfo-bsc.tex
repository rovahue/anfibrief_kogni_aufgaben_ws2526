Wie du dem folgenden Stundenplan entnehmen kannst, enthält das Medizininformatik-Studium im ersten Semester neben der Informatik- und Mathematikvorlesung auch eine gute Portion Humanbiologie, außerdem dürft ihr Euch mit grundlegenden Problemen der IT im Gesundheitswesen beschäftigen sowie Euch grundlegende medizinische Fachbegriffe aneignen.\\
% Bitte beachte, dass die Veranstaltungszeiten des Moduls \textit{Medizinische Terminologie} zu Redaktionsschluss noch nicht vorlag und daher im untenstehenden Plan fehlt! \\
\noindent\makebox[\textwidth][c]{%
	\setlength{\fboxrule}{4pt}
	\fcolorbox{red}{white}{
		\begin{minipage}[t]{
			%\textwidth}\textbf{Achtung!} Aufgrund der aktuellen Lage bezüglich COVID-19 können sich die Vorlesungstermine für dieses Semester noch ändern. Schau am besten auf Alma (\url{https://alma.uni-tuebingen.de}), ob die Termine dort geupdatet wurden.
			\textwidth}\textbf{Achtung!} Die Daten für die Vorlesungstermine können sich noch ändern. Schau am besten auf Alma (\url{https://alma.uni-tuebingen.de}), ob die Termine dort geupdatet wurden.
		\end{minipage}}}

\begin{minipage}{\textwidth}
    \footnotesize
\begin{center}
\begin{tabular}{|c|c|c|c|c|c|}
	\hline
	Zeit    & Montag       	& Dienstag    	& Mittwoch                  	& Donnerstag		& Freitag   \\\hline\hline
	08 – 09 & Mathematik I 	&             	& Mathematik I              	&			&			\\\cline{1-1}\cline{3-3}\cline{5-6}
	09 – 10 & (\Matheprof) 	&             	& (\Matheprof)              	&   			&			\\\hline
	10 – 11 &              	&             	& Grundlagen der Medizininf.	&			&			\\\cline{1-3}\cline{5-6}
	11 – 12 &              	&             	& 						       	&			&			\\\hline
	12 – 13 &              	&             	&                           	&			&			\\\hline
	13 – 14 &              	&             	&                           	&			&			\\\hline
	14 – 15 &              	& Informatik I	&                           	& Informatik I		&			\\\cline{1-2}\cline{4-4}\cline{6-6}
	15 – 16 &              	& (\Infoprof) 	&                           	& (\Infoprof)		&			\\\hline
	16 – 17 & Humanbiologie	& Humanbiologie &                           	&			&			\\\cline{1-1}\cline{4-6}
	17 – 18 &              	&             	&                           	& Med. Terminologie	&			\\\cline{1-4}\cline{6-6}
 	18 – 19 &              	&             	&                           	& 			&			\\\hline
\end{tabular}
    ~\\
% \scriptsize %\\
%HSM = gr. Hörsaal Medizinische Klinik (Uni-Kliniken Berg); H2C14 = Morgenstelle, H-Bau Erdgeschoss\\
%1: beginnt immer s.t., findet erst ab der zweiten Vorlesungswoche statt\\
\end{center}
\end{minipage}

Dieser Plan gilt für das erste Semester Medizininformatik. Es kommen noch Übungen für die Vorlesungen
Informatik I und Mathematik I dazu. Die Zeiten für die Übungsgruppen werden innerhalb der ersten Woche in den Vorlesungen bekannt gegeben. \\ \\
